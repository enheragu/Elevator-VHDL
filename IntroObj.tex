

\subsection{Objetivos:}
    \begin{itemize}
        \item Diseño de la lógica de control de un ascensor.
        \item Implementación y validación de dicho diseño en lenguaje VHDL.
        \item Simulación de cada entidad implementada para verificar su funcionamiento mediante un testbench en código VHDL.
        \item Sintetizado sobre la tarjeta SPARTAN-3 Starter Kit del laboratorio
    \end{itemize}
    
\subsection{Funcionamiento general del Ascensor y requisitos:}

   El trabajo consiste en diseñar el controlador de un ascensor único en una vivienda.
    \begin{itemize}
        \item Requisitos mínimos:
        \begin{itemize}
    		    \item El ascensor debe tener un mínimo de 4 pisos.
    		    \item Las entradas al circuito serán, por un lado, el piso al que el usuario desea ir y el piso en el que está el ascensor en un momento dado.
    		    \item Hay 2 salidas, la del motor (2 bits) y la de la puerta (1 bit).
    		    \item El ascensor debe ir al piso indicado, cuando llegue abrirá las puertas que permanecerán así hasta que se reciba otra llamada.
    		    \item Si mientras que el ascensor se mueve, se pulsan otros botones no se debe hacer caso.
    		    \item Utilizar los LEDs y los Displays para visualizar la información.
    	\end{itemize}
        \item Funcionalidad extra
        \begin{itemize}
            \item [1-] Añadimos dos salidas. 
            \begin{itemize}
        		    \item Piso al que el usuario desea ir.
                \item Piso en el que está el ascensor.
            \end{itemize}  
            Mostrando ambas por dos dígitos distintos del Display.
            \item [2-] Aumentamos el número de pisos que puede almacenar simultáneamente el programa a tres, debe ir a cada uno de ellos en el mismo orden que fueron introducidos.
            \begin{itemize}
        		    \item Se muestran todos los pisos almacenados.
      	  	    \item Se mueva o no el ascensor, se hace caso de los botones pulsados mientras alguna de las tres variables esté "vacía".
    	    \end{itemize}
	    \end{itemize}
        \item Puertos usados:
        \begin{itemize}
     	   	\item Pulsadores de la FPGA, se usan para la entrada del piso en el que está el ascensor en un momento dado, representan finales de carrera (uno por planta).
    		    \item Cuatro Interruptores, empleados para introducir el piso al que el usuario desea ir, representan los botones del ascensor (uno por planta).
    		    \item Cuatro dígitos del Display, dedicados a mostrar el estado del motor, el estado de la puerta el piso en el que está el ascensor y el piso al que va el ascensor.
    		    \item LEDs de la FPGA, utilizados para representar los dos pisos almacenados a los que posteriormente debe ir el ascensor.
    	\end{itemize}
	\end{itemize}

\subsection{Placa Spartan-3 Starter Kit Board} \label{subsection:Spartan-3}
    
    Para la implementación de este proyecto y si realización sobre un soporte físico real se utilizará la placa Spartan-3 Starter Kit Board de Xilinx, que incluye la FPGA así como otros componentes que nos vendrán bien a la hora de simular el comportamiento del ascensor.

    En las figuras que se muestran a continuación se pueden ver la distribución de los componentes del Kit:

    \begin{figure}[H]
            \centering
            \includegraphics[width = 0.8\textwidth ]{Spartan3TopSide}
            \caption{Placa Spartan-3 (parte superior)}
            \label{fig:Spartan3TopSide}
    \end{figure}

    \begin{figure}[H]
            \centering
            \includegraphics[width = 0.8\textwidth ]{Spartan3BottomSide}
            \caption{Placa Spartan-3 (parte inferior)}
            \label{fig:Spartan3BottomSide}
    \end{figure}

    Nos centraremos en los componentes más importantes que utilizaresmos para la realización de este proyecto:

    \begin{itemize}
        \item [1.] Xilinx Spartan-3 XC3S200 FPGA  (encapsulado XC3S200FT256). Compuesta por 200000 puertas.
        \item [2.] 2Mbit Xilinx XCF02S Platform Flash.
        \item [4.] 1M-byte of Fast Asynchronous SRAM.
        \item [10.] Cuatro displays LED de 7 segmentos.
        \item [11.] Ocho interruptores.
        %\item [12.] Ocho salidas LED independientes.
        \item [13.] Cuatro pulsadores.
        \item [14.] Cristal osculador (CLK) de 50MHz.
        \item [18.] LED indicador de que la FPGA ha sido configurada correctamente

    \end{itemize}

    Las salidas físicas para los \textbf{displays de 7 segmentos} se encuentran en los pines que se pueden ver en la siguiente tabla. Los displays se encuentran representados con el número 10 en la figura (\ref{fig:Spartan3TopSide}) del apartado (\ref{subsection:Spartan-3}). \\ 

    Como se puede ver en la figura los 4 displays comparten 8 pines de control, para elegir un display u otro están los Ánodos de control, en la tabla (\ref{tab:anodoControl}). \\ 

    Es importante tener en cuenta que los cuatro display de 7 segmentos comparten la entrada de 7 bits que codifica el caracter; se elige por que display se muestra dicha información sacando un valor de alto nivel por el ánodo de control correspondiente. \\ 

    \begin{table}[H]
            \centering
            \begin{tabular}{|c|c|c|c|c|c|c|c|c|}
                \hline
                \rowcolor[rgb]{0.21,0.69,0.87}\multicolumn{9}{|c|}{  \textbf{ {Salidas físicas Display 7 segmentos}}} \\
                \hline \hline
                \textbf{  Segmento  } & A & B & C & D & E & F & G & DP \\ 
                \hline
                \textbf{  FPGA Pin  }  & E14 & G13 & N15 & P15 & R16 & F13 & N16 & P16 \\ 
                \hline
                \multicolumn{9}{|c|}{\includegraphics[width = 0.8\textwidth ]{Spartan3-7segment}}\\
                \hline
                 
            \end{tabular}
        \caption{ Salidas físicas de los displays en la Spartan-3 }
        \label{tab:tablaSalidas7Segmentos}
    \end{table}

    \begin{table}[H]
            \centering
            \begin{tabular}{|c|c|c|c|c|}
                \hline
                \rowcolor[rgb]{0.21,0.69,0.87}\multicolumn{5}{|c|}{  \textbf{ {Ánodos de control}}} \\
                \hline \hline
                \textbf{  Anodo Control  } & AN3 & AN2 & AN1 & AN0  \\
                \hline
                \textbf{  FPGA Pin  } & E13 & F14 & G14 & D14  \\
                \hline
                 
            \end{tabular}
        \caption{ Ánodos control (activos a nivel bajo) para los display de 7 segmentos }
        \label{tab:anodoControl}
    \end{table}

    Las entradas correspondientes a los \textbf{interruptores} se pueden ver en la siguiente tabla. Estos interruptores se encuentran representados con el número 11 en la figura (\ref{fig:Spartan3TopSide}) del apartado (\ref{subsection:Spartan-3}).

    \begin{table}[H]
            \centering
            \begin{tabular}{|c|c|c|c|c|c|c|c|c|}
                \hline
                \rowcolor[rgb]{0.21,0.69,0.87}\multicolumn{9}{|c|}{  \textbf{ {Entradas Interruptores}}} \\
                \hline \hline
                \textbf{  Interruptor  } & SW7 & SW6 & SW5 & SW4 & SW3 & SW2 & SW1 & SW0 \\
                \hline
                \textbf{  FPGA Pin  } & K13 & K14 & J13 & J14 & H13 & H14 & G12 & F12 \\
                \hline
                 
            \end{tabular}
        \caption{ Entradas físicas de los interruptores en la Spartan-3 }
        \label{tab:tablaEntradasInterruptores}
    \end{table}

    Los pines de entrada correspondientes a los \textbf{pulsadores} se pueden ver en la siguiente tabla. Estos componentes se corresponden con los representados con el número 13 en la figura (\ref{fig:Spartan3TopSide}) del apartado (\ref{subsection:Spartan-3}).

    \begin{table}[H]
            \centering
            \begin{tabular}{|c|c|c|c|c|}
                \hline
                \rowcolor[rgb]{0.21,0.69,0.87}\multicolumn{5}{|c|}{  \textbf{ {Entradas Pulsadores}}} \\
                \hline \hline
                \textbf{  Pulsador  } & BTN3 (Reset) & BTN2 & BTN1 & BTN0  \\
                \hline
                \textbf{  FPGA Pin  } & L14 & L13 & M14 & M13  \\
                \hline
                 
            \end{tabular}
        \caption{ Entradas físicas de los pulsadores en la Spartan-3 }
        \label{tab:tablaEntradasPulsadores}
    \end{table}

%    Por ahora no utilizamos los LEDs, queda comentado por si volvemos a usarlos.
%
%    Como se ha dicho en el apartado (\ref{subsection:Spartan-3}) la placa incorpora ocho LEDs, representados con el número 12 en la figura (\ref{fig:Spartan3TopSide}). A continuación se muestra la correlación de cada LED con la patilla de la FPGA:

%   \begin{table}[H]
%            \centering
%            \begin{tabular}{|c|c|c|c|c|c|c|c|c|}
%                \hline
%                \rowcolor[rgb]{0.21,0.69,0.87}\multicolumn{9}{|c|}{  \textbf{ {Salidas LED}}} \\
%                \hline \hline
%                \textbf{  LED  } & LD7 & LD6 & LD5 & LD4 & LD3 & LD2 & LD1 & LD0 \\
%                \hline
%                \textbf{  FPGA Pin  } & P11 & P12 & N12 & P13 & N14 & L12 & P14 & K12  \\
%                \hline
%                 
%            \end{tabular}
%        \caption{ Salidas físicas de los LEDs en la Spartan-3 }
%        \label{tab:tablaSalidasLED}
%    \end{table}

\subsection{Estructura de la memoria e información útil}

    En los siguientes apartados de la memoria se puede encontrar la explicación detallada del funcionamiento y codificación de la lógica del ascensor descrito, concretamente:
    \begin{itemize}
        \item Apartado \ref{section:DiagBloques}: Se detalla el funcionamiento interno de cada entidad o arquitectura así como su interfaz. A su vez se describe la relación entre las diferentes entidades
        \item Apartado \ref{section:Codigo}: Se adjunta la programación de cada entidad o arquitectura así como su correspondiente testbech.
        \item Apartado \ref{section:PruebasYResultados}: Se comentan los aspectos prácticos de como se ha cargado esta información en la FPGA así como los resultados obtenidos.
        \item Apéndice \ref{app:codEntSal}: Se adjuntan las tablas donde se especifica la codificación que se ha utilizado para el funcionamiento interno del ascensor.
    \end{itemize}
    
    Todo el proyecto, tanto el documento en código \LaTeX\  como los ficheros VHDL se pueden encontrar en Github \faGithub\ en el siguiente enlace: https://github.com/enheragu/Elevator-VHDL
