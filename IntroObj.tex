

\subsection{Objetivos:}
    \begin{itemize}
        \item Diseño de la lógica de control de un ascensor.
        \item Implementación y validación de dicho diseño en lenguaje VHDL.
        \item Simulación de cada entidad implementada para verificar su funcionamiento mediante un testbench en código VHDL.
        \item Sintetizado sobre la tarjeta SPARTAN-3 Starter Kit del laboratorio
    \end{itemize}
    
\subsection{Funcionamiento general del Ascensor y requisitos:}

\subsection{Estructura de la memoria e información útil}

    En los siguientes apartados de la memoria se puede encontrar la explicación detallada del funcionamiento y codificación de la lógica del ascensor descrito, concretamente:
    \begin{itemize}
        \item Apartado \ref{section:DiagBloques}: Se detalla el funcionamiento interno de cada entidad o arquitectura así como su interfaz. A su vez se describe la relación entre las diferentes entidades
        \item Apartado \ref{section:Codigo}: Se adjunta la programación de cada entidad o arquitectura así como su correspondiente testbech.
        \item Apéndice \ref{app:codEntSal}: Se adjuntan las tablas donde se especifica la codificación que se ha utilizado para el funcionamiento interno del ascensor.
    \end{itemize}
    
    Todo el proyecto, tanto el documento en código \LaTeX\  como los ficheros VHDL se pueden encontrar en Github \faGithub\ en el siguiente enlace: https://github.com/enheragu/Elevator-VHDL
