\documentclass[a4paper,11pt]{article} %openright - los capítulos empiezan en páginas impares; 10pt tamaño de texto
\usepackage[spanish]{babel}
\usepackage[utf8]{inputenc} %para trabajar con símbolos no anglosajones

\usepackage[title,titletoc,toc]{appendix}
\usepackage{textcomp}
\usepackage{graphicx}
\usepackage{eso-pic}
\usepackage{transparent}
\usepackage{enumerate}
\usepackage{fancyhdr}
\usepackage{graphicx}
\usepackage{subfigure}
\usepackage{float}
\usepackage{colortbl}
\usepackage{amsmath}
\usepackage{gensymb}
\usepackage{multirow}
\usepackage{subfigure}
\usepackage{minted}
\usepackage{anysize}
\usepackage{fontawesome}

\usepackage{longtable}
%
% Hace una marca de agua
%

%\usepackage{background}
%\SetBgContents{ENTREGA PROVISIONAL}
%\SetBgScale{5} %% Escala del texto
%\SetBgColor{black!10!white} %% Tonalidad



\marginsize{3cm}{2cm}{2cm}{2cm} % Controla los márgenes {izquierda}{derecha}{arriba}{abajo}. 

\graphicspath{{Pictures/}} %path de los gráficos

%
% A continuación se define la portada
%

\newcommand\BackgroundPic{%
	\put(0,0){%
		\parbox[b][\paperheight]{\paperwidth}{%
			\vfill
			\centering
			\includegraphics[width=\paperwidth,height=\paperheight,%
			keepaspectratio]{PortadaAzul.pdf}%
			\vfill
		}}}
%
% A continuación está el encabezado y pie de página
%
		
\pagestyle{fancy}
\fancyfoot[RE,RO]{Página \thepage}
\fancyfoot[C]{\ }
\fancyfoot[LE,LO]{\textcolor{gray}{SED. Proyecto Ascensor}}
\fancyhead[RE,RO]{Grupo N\textdegree 4}
\fancyhead[LE,LO]{\ }

%
% A continuación empieza el documento
%

\begin{document}

    
    %
    % A continuación empieza la portada del documento
    %

	\begin{titlepage}
		\pagestyle{empty}
		\AddToShipoutPicture*{\BackgroundPic}
		\begin{center}
			\Huge \textbf{Proyecto final de asignatura:\\ \textsc{“Ascensor”}}
			\vspace{5cm}
			
			\Large \textsc{Dpto. Electrónica, Automática e Informática	Industrial} \\ 
			
			\large \textsc{ETSIDI - UPM}
		\end{center}
		\vspace{8cm}
		
		\hspace{1cm}\large Asignatura de Sistemas Electrónicos Digitales
		
		\hspace{1cm}\large Curso 2016-2017\\
		
		
		\hspace{1cm}\large \textbf{Arancha Canelo Martin} \hspace{1.65cm} N\textdegree 50654 \\ 
		
		\hspace{1cm}\large \textbf{Enrique Heredia Aguado} \hspace{1.4cm} N\textdegree 50688 \\ 
		
		\hspace{1cm}\large \textbf{Alberto Beneit Benita} \hspace{1.95cm} N\textdegree 51104
		\ClearShipoutPicture
		\thispagestyle{empty}
	\end{titlepage}
	
	\pagestyle{fancy}
	
	%\addtocontents{toc}{\hspace{-7.5mm} \textbf{Contenidos:}}
	\addtocontents{toc}{\hfill \textbf{Página:} \par}
	\addtocontents{toc}{\vspace{-2mm} \hspace{-7.5mm} \hrule \par}
	\tableofcontents
	\newpage

    %
    % A continuación empieza el contenido del documento
    %
    \section{Introducción:} \label{section:Introduccion}
        

\subsection{Objetivos:}
    \begin{itemize}
        \item Diseño de la lógica de control de un ascensor.
        \item Implementación y validación de dicho diseño en lenguaje VHDL.
        \item Simulación de cada entidad implementada para verificar su funcionamiento mediante un testbench en código VHDL.
        \item Sintetizado sobre la tarjeta SPARTAN-3 Starter Kit del laboratorio
    \end{itemize}
    
\subsection{Funcionamiento general del Ascensor y requisitos:}

\subsection{Estructura de la memoria e información útil}

    En los siguientes apartados de la memoria se puede encontrar la explicación detallada del funcionamiento y codificación de la lógica del ascensor descrito, concretamente:
    \begin{itemize}
        \item Apartado \ref{section:DiagBloques}: Se detalla el funcionamiento interno de cada entidad o arquitectura así como su interfaz. A su vez se describe la relación entre las diferentes entidades
        \item Apartado \ref{section:Codigo}: Se adjunta la programación de cada entidad o arquitectura así como su correspondiente testbech.
        \item Apéndice \ref{app:codEntSal}: Se adjuntan las tablas donde se especifica la codificación que se ha utilizado para el funcionamiento interno del ascensor.
    \end{itemize}
    
    Todo el proyecto, tanto el documento en código \LaTeX\  como los ficheros VHDL se pueden encontrar en Github \faGithub\ en el siguiente enlace: https://github.com/enheragu/Elevator-VHDL
 %importa el fichero IntroObj.tex con la introducción, objetivos y explicación general del proyecto
        
    \section{Diagrama de bloques:} \label{section:DiagBloques}
    
        \subsection{Entidad \textit{Acensor}:}
    
    La interfaz, entradas y salidas, de este bloque se puede ver su representación en la Figura \ref{fig:EntidadIAscensor}:
    
    \begin{figure}[H]
		    \centering
		    \includegraphics[width = 0.8\textwidth ]{EntidadAscensor}
		    \caption{Diagrama de la Interfaz de la Entidad Ascensor}
		    \label{fig:EntidadIAscensor}
	\end{figure}
	
	Esta es una entidad de alto nivel que encapsula todo el funcionamiento del ascensor. Se puede ver el diagrama de bloques interno de esta entidad en la siguiente figura: 
	
	\begin{figure}[H]
		    \centering
		    \includegraphics[width = .85\textwidth ]{EntidadAscensorEntidades}
		    \caption{Diagrama interno de la entidad Ascensor}
		    \label{fig:EntidadIAscensorEntidades}
	\end{figure}
	
	Como se puede ver en la figura anterior se han encapsulado los bloques en diferentes entidades en función de su finalidad, estas entidades se pueden ver a continuación. En las representaciones de estas entidades se verá que hay ciertos bloques repetidos, se trata del mismo bloque utilizado dos veces, no de bloques diferentes, que se han representado por duplicado para mayor claridad. \\ 

	Se adelanta el esquema del funcionamiento general del ascensor (obviando las entidades generales):

	\begin{figure}[H]
		    \centering
		    \includegraphics[width = .85\textwidth ]{EntidadAscensorInterior}
		    \caption{Diagrama funcionamiento del Ascensor}
		    \label{fig:EntidadIAscensorInterior}
	\end{figure}

\subsection{Entidad \textit{Interfaz Entrada}:}
	Esta entidad se encarga de gestionar los datos de las entradas para adaptarlos al funcionamiento interno de nuestro sistema, en este caso se encarga de codificar y modular correctamente las señales de los sensores del piso al que voy y del piso en que estoy. \\ 

	Como se ve en la figura (\ref{fig:EntidadesAscensorE1}), engloba dos bloques, que se desgran en los siguientes apartados.
	\begin{figure}[H]
		    \centering
		    \includegraphics[width = .85\textwidth ]{EntidadAscensorInterior(E1)}
		    \caption{Representación de la Entidad Interfaz Entrada}
		    \label{fig:EntidadesAscensorE1}
	\end{figure}

\subsection{Entidad \textit{Control Ascensor}:}
	Esta entidad se encarga de gestionar el funcionamiento del ascensor propiamente dicho comparando las lecturas de los sensores y decidiendo que deben hacer los actuadores, el motor y la puerta.  \\ 

	Como se ve en la figura (\ref{fig:EntidadesAscensorE3}), engloba tres bloques, que se explican más adelante.
	\begin{figure}[H]
		    \centering
		    \includegraphics[width = .85\textwidth ]{EntidadAscensorInterior(E3)}
		    \caption{Representación de la Entidad Control Ascensor}
		    \label{fig:EntidadesAscensorE3}
	\end{figure}

	Como se puede observar el bloque Codigicador Binario-Decimal se utiliza dos veces.

\subsection{Entidad \textit{Visualizacion}:}	
	Esta entidad se encarga de gestionar el funcionamiento del ascensor propiamente dicho comparando las lecturas de los sensores y decidiendo que deben hacer los actuadores, el motor y la puerta.  \\ 

	Como se ve en la figura (\ref{fig:EntidadesAscensorE2}), engloba cuatro bloques bloques, que se detallan en apartados posteriores.
	\begin{figure}[H]
		    \centering
		    \includegraphics[width = .85\textwidth ]{EntidadAscensorInterior(E2)}
		    \caption{Representación de la Entidad Visualizacion}
		    \label{fig:EntidadesAscensorE2}
	\end{figure}

	Como se puede observar el bloque Decodificador7s y el bloque DivisorFrecuencia se utilizan dos veces cada uno.


\subsection{Bloque \textit{Decodificador a 7 segmentos}:}
    Este bloque es el encargado de traducir el piso en el que se encuentra el ascensor y el piso objetivo para poder mostrarlo en el display de 7 segmentos. Para ello es importante recordar el Cuadro (\ref{tab:tabla1ApendiceA}) del Apéndice \ref{app:codEntSal} donde podemos ver como se codifica internamente el número de piso. \\ 
    
    La salida que se verá en el display para cada caso se puede ver en la figura siguiente:
    
    \begin{figure}[H]
		    \centering
		    \includegraphics[width = 1\textwidth ]{displays7s}
		    \caption{Salida en los displays de 7 segmentos}
		    \label{fig:displays7s}
	\end{figure}
    
    La interfaz, entradas y salidas, de este bloque se puede ver su representación en la Figura \ref{fig:BloqueDecodificador7seg}:
    
    \begin{figure}[H]
		    \centering
		    \includegraphics[width = 0.7\textwidth ]{BloqueDecodificador}
		    \caption{Diagrama Interfaz Bloque Decodificador a 7 segmentos}
		    \label{fig:BloqueDecodificador7seg}
	\end{figure}
\subsection{Bloque \textit{Divisor de frecuencia}:}

\subsection{Bloque \textit{PistoActual}:}
    Como se ha dicho anteriormente en cada piso hay un final de carrera que detecta el paso del ascensor. El prpósito de este bloque es el de filtrar dicha entrada de 4 bits. En este caso lo que interesa es saber en que piso estoy o en que piso he estado por última vez. Cuando el ascensor se encuentra entre dos pisos la entrada de los sensores será \textit{0000}, este bloque lo que hará será mantener en la salida del mismo el último piso por el que haya pasado el ascensor. \\ 
    
    Como se puede apreciar en el siguiente diagrama este bloque tiene una entrada, un vector de 4 bits (los finales de carrera de cada piso) y una salida, también de 4 bits, codificando el piso en el que se encuentra actualmente. \\ 
    
    Se puede consultar dicha codificación en el Cuadro (\ref{tab:tabla1ApendiceA}) del Apéndice \ref{app:codEntSal}. \\ 
    
    La interfaz, entradas y salidas, de este bloque se puede ver su representación en la Figura \ref{fig:BloquePisoActual}:
    
    \begin{figure}[H]
		    \centering
		    \hspace*{-1.8cm}
		    \includegraphics[width = 0.6\textwidth ]{BloquePisoActual}
		    \caption{Diagrama Bloque PisoActual}
		    \label{fig:BloquePisoActual}
	\end{figure}
	
	Se puede consultar el código VHDL de este módulo en el Apartado \ref{code:PisoActual} así como el código de su testbench correspondiente en el Apartado \ref{code:PisoActual_tb}.

\subsection{Bloque \textit{Bloqueador PisoVoy}:}
    
    La interfaz, entradas y salidas, de este bloque se puede ver su representación en la Figura \ref{fig:BloqueBloqueadorPisoVoy}:
    
    
    \begin{figure}[H]
		    \centering
		    \hspace*{-1.8cm}
		    \includegraphics[width = 0.6\textwidth ]{BloqueBloqueadorPisoVoy}
		    \caption{Diagrama Bloque Bloqueador PisoVoy}
		    \label{fig:BloqueBloqueadorPisoVoy}
	\end{figure}
	
\subsection{Código Bloque \textit{Decodificador Binario a Decimal}:} 
	En este bloque se traduce la señal binaria que codifica tanto el piso actual como el piso de destino a decimal para su posterior comparación.
	Se puede consultar dicha codificación en el Cuadro (\ref{tab:tabla1ApendiceA}) del Apéndice \ref{app:codEntSal}. \\ 
	
\subsection{Bloque \textit{Comparador}:}

    La interfaz, entradas y salidas, de este bloque se puede ver su representación en la Figura \ref{fig:BloqueComparador}:
    
    \begin{figure}[H]
		    \centering
		    \includegraphics[width = 0.6\textwidth ]{BloqueComparador}
		    \caption{Diagrama Bloque Comparador}
		    \label{fig:BloqueComparador}
	\end{figure}



\subsection{Bloque \textit{Controlador Motor} y \textit{Controlador Puerta}:}
    El bloque Controlador Motor se encarga de controlar el motor. En la entrada recibe dos bits con la codificación que se muestra en el Cuadro (\ref{tab:tabla2ApendiceA}) del Apéndice \ref{app:codEntSal} y los gestiona para modelar y simular con los recursos de la placa el movimiento del mismo.

    El bloque Controlador Puerta se encarga del control de la puerta del ascensor. En la entrada recibe un bit con la codificación que se muestra en el Cuadro (\ref{tab:tabla3ApendiceA}) del Apéndice \ref{app:codEntSal} que indica si la puerta está abierta o cerrada; este bloque actua como interfaz entre esa codificación y el modelo real, en este caso se simulará con los recursos disponibles en la placa.

    En este caso no tenemos ni puerta ni motor físico así que se ha decidido sacar la "reacción" que tendrían estos elementos por dos de los display de 7 segmentos de que se dispone, de esta forma se podrá ver en los displays los siguientes caracteres:

	\begin{table}[H]
        \centering
			\begin{tabular}{|c|c||c|c|}
				\hline
				\rowcolor[rgb]{0.21,0.69,0.87}\multicolumn{2}{|c|}{  \textbf{ {Funcionamiento Motor}}} & \multicolumn{2}{|c|}{  \textbf{ {Funcionamiento Puerta}}} \\
				\hline \hline
				 & \textbf{ Codificación Interna }  & \textbf{ Codificación Interna }  \\
				\hline
				Subiendo & \includegraphics[width = 0.8\textwidth ]{CodCaracteres7s/subida}} & Abierta & \includegraphics[width = 0.8\textwidth ]{CodCaracteres7s/abierta}}  \\
				\hline
				Bajando & \includegraphics[width = 0.8\textwidth ]{CodCaracteres7s/bajada}} & Cerrada & \includegraphics[width = 0.8\textwidth ]{CodCaracteres7s/cerrada}}  \\
				\hline
				Parado & \includegraphics[width = 0.8\textwidth ]{parado}} & - & - \\
				\hline				 
			\end{tabular}
			\caption{ Salida en displays 7 segmentos para el control del motor y de la puerta }
			\label{tab:tabla2ApendiceA}
	\end{table}
    
	Este bloque consta de un bloque para la codificación de ambos displays en serie con los respectivos divisores de frecuencia para cada display. %importa el fichero DiagramaBloques.tex
        
    \section{Código:} \label{section:Codigo}
    
        \subsection{Código Bloque \textit{Decodificador a 7 segmentos}:} \label{code:Decodificador7s}
	\inputminted[frame=lines,fontsize=\footnotesize,linenos]{vhdl}{CodeFiles/Decodificador7s.vhd}

\subsection{Código Bloque \textit{Decodificador a 7 segmentos (testbench)}:} \label{code:Decodificador7s_tb}
	
	\inputminted[frame=lines,fontsize=\footnotesize,linenos]{vhdl}{CodeFiles/Decodificador7s_tb.vhd}
	
	Utilizando este testbench se obtiene el comportamiento que se puede ver en la Figura (\ref{fig:SimulacionDecodificador7s}):

    \begin{figure}[H]
		    \centering
		    \includegraphics[width = 0.8\textwidth ]{SimulacionDecodificador7s}
		    \caption{Simulación del testbench del bloque Decodificador7s}
		    \label{fig:SimulacionDecodificador7s}
	\end{figure}

\subsection{Código Bloque \textit{Divisor de frecuencia}:} \label{code:DivisorFrecuencia}

\subsection{Código Bloque \textit{Divisor de frecuencia (testbench)}:} \label{code:DivisorFrecuencia_Tb}

\subsection{Código Bloque \textit{PistoActual}:} \label{code:PisoActual}
    \inputminted[frame=lines,fontsize=\footnotesize,linenos]{vhdl}{CodeFiles/PisoActual.vhd}
    
    Como se puede ver en la Figura (\ref{fig:BloquePisoActualOK}) el esquema obtenido una vez programado y sintetizado se corresponde con el que se pretendía.
    \begin{figure}[H]
		    \centering
		    \includegraphics[width = 0.6\textwidth ]{BloquePisoActualOK}
		    \caption{Esquema exterior del bloque Piso Actual}
		    \label{fig:BloquePisoActualOK}
	\end{figure}
    Además podemos ver en la Figura (\ref{fig:BloquePisoActualImplementacion}) como se compone internamente el bloque, como se codifica en hardware esta utilidad:
    \begin{figure}[H]
		    \centering
		    \includegraphics[width = 0.9\textwidth ]{BloquePisoActualImplementacion}
		    \caption{Esquema interno del bloque Piso Actual}
		    \label{fig:BloquePisoActualImplementacion}
	\end{figure}

\subsection{Código Bloque \textit{PistoActual (testbench)}:} \label{code:PisoActual_tb}
    \inputminted[frame=lines,fontsize=\footnotesize,linenos]{vhdl}{CodeFiles/PisoActual_tb.vhd}
    
    Utilizando este testbench se obtiene el comportamiento que se puede ver en la Figura (\ref{fig:SimulacionPisoActual}):

    \begin{figure}[H]
		    \centering
		    \includegraphics[width = 0.8\textwidth ]{SimulacionPisoActual}
		    \caption{Simulación del testbench del bloque PisoActual}
		    \label{fig:SimulacionPisoActual}
	\end{figure}


\subsection{Código Bloque \textit{Bloqueador PisoVoy}:} \label{code:BloqueadorpisoVoy}	
    \inputminted[frame=lines,fontsize=\footnotesize,linenos]{vhdl}{CodeFiles/BloqueadorPisoVoy.vhd}
	Como se puede ver en la Figura (\ref{fig:BloqueBloqueadorPisoVoyOK}) el esquema obtenido una vez programado y sintetizado se corresponde con el que se pretendía.
    \begin{figure}[H]
		    \centering
		    \includegraphics[width = 0.6\textwidth ]{BloqueBloqueadorPisoVoyOK}
		    \caption{Esquema exterior del bloque Bloqueador PisoVoy}
		    \label{fig:BloqueBloqueadorPisoVoyOK}
	\end{figure}
    Además podemos ver en la Figura (\ref{fig:BloqueBloqueadorPisoVoyImplementacion}) como se compone internamente el bloque, como se codifica en hardware esta utilidad:
    \begin{figure}[H]
		    \centering
		    \includegraphics[width = 0.9\textwidth ]{BloqueBloqueadorPisoVoyImplementacion}
		    \caption{Esquema interno del bloque Bloqueador PisoVoy}
		    \label{fig:BloqueBloqueadorPisoVoyImplementacion}
	\end{figure}

\subsection{Código Bloque \textit{Bloqueador PisoVoy (testbench)}:} \label{code:BloqueadorpisoVoy_tb}
    \inputminted[frame=lines,fontsize=\footnotesize,linenos]{vhdl}{CodeFiles/BloqueadorPisoVoy_tb.vhd}

    Utilizando este testbench se obtiene el comportamiento que se puede ver en la Figura (\ref{fig:SimulacionBloqueadorPisoVoy}):

    \begin{figure}[H]
		    \centering
		    \includegraphics[width = 0.8\textwidth ]{SimulacionBloqueadorPisoVoy}
		    \caption{Simulación del testbench del bloque BloqueadorPisoVoy}
		    \label{fig:SimulacionBloqueadorPisoVoy}
	\end{figure}

\subsection{Código Bloque \textit{Decodificador Binario a Entero}:} \label{code:DecodificadorBinarioEntero}
    \inputminted[frame=lines,fontsize=\footnotesize,linenos]{vhdl}{CodeFiles/DecodificadorBinarioEntero.vhd}
    
    Como se puede ver en la Figura (\ref{fig:BloqueDecodificadorBinarioEnteroOK}) el esquema obtenido una vez programado y sintetizado se corresponde con el que se pretendía.
    \begin{figure}[H]
		    \centering
		    \includegraphics[width = 0.6\textwidth ]{BloqueDecodificadorBinarioEnteroOK}
		    \caption{Esquema exterior del bloque Decodificador Binario-Entero}
		    \label{fig:BloqueDecodificadorBinarioEnteroOK}
	\end{figure}
    Además podemos ver en la Figura (\ref{fig:BloqueDecodificadorBinarioEnteroImplementacion}) como se compone internamente el bloque, como se codifica en hardware esta utilidad:
    \begin{figure}[H]
		    \centering
		    \includegraphics[width = 0.9\textwidth ]{BloqueDecodificadorBinarioEnteroImplementacion}
		    \caption{Esquema interno del bloque Decodificador Binario-Entero}
		    \label{fig:BloqueDecodificadorBinarioEnteroImplementacion}
	\end{figure}
    
\subsection{Código Bloque \textit{Decodificador Binario a Entero (testbench)}:} \label{code:DecodificadorBinarioEntero_tb}
    \inputminted[frame=lines,fontsize=\footnotesize,linenos]{vhdl}{CodeFiles/DecodificadorBinarioEntero_tb.vhd}

    Utilizando este testbench se obtiene el comportamiento que se puede ver en la Figura (\ref{fig:SimulacionDecodificadorBinarioEntero}):

    \begin{figure}[H]
		    \centering
		    \includegraphics[width = 0.8\textwidth ]{SimulacionDecodificadorBinarioEntero}
		    \caption{Simulación del testbench del bloque DecodificadorBinarioEntero}
		    \label{fig:SimulacionDecodificadorBinarioEntero}
	\end{figure}

\subsection{Código Bloque \textit{Comparador}:} \label{code:Comparador}
    \inputminted[frame=lines,fontsize=\footnotesize,linenos]{vhdl}{CodeFiles/Comparador.vhd}	

	Como se puede ver en la Figura (\ref{fig:BloqueComparadorOK}) el esquema obtenido una vez programado y sintetizado se corresponde con el que se pretendía.
    \begin{figure}[H]
		    \centering
		    \includegraphics[width = 0.6\textwidth ]{BloqueComparadorOK}
		    \caption{Esquema exterior del Bloque Comparador}
		    \label{fig:BloqueComparadorOK}
	\end{figure}
    Además podemos ver en la Figura (\ref{fig:BloqueComparadorImplementacion}) como se compone internamente el bloque, como se codifica en hardware esta utilidad:
    \begin{figure}[H]
		    \centering
		    \includegraphics[width = 0.9\textwidth ]{BloqueComparadorImplementacion}
		    \caption{Esquema interno del Bloque Comparador}
		    \label{fig:BloqueComparadorImplementacion}
	\end{figure}

\subsection{Código Bloque \textit{Comparador (testbench)}:} \label{code:Comparador_tb}
    \inputminted[frame=lines,fontsize=\footnotesize,linenos]{vhdl}{CodeFiles/Comparador_tb.vhd}

    Utilizando este testbench se obtiene el comportamiento que se puede ver en la Figura (\ref{fig:SimulacionComparador}):

    \begin{figure}[H]
		    \centering
		    \includegraphics[width = 0.8\textwidth ]{SimulacionComparador}
		    \caption{Simulación del testbench del bloque Comparador}
		    \label{fig:SimulacionComparador}
	\end{figure}

\subsection{Código Bloque \textit{ Simulacion Motor Puerta}:} \label{code:MotorPuerta}
	\inputminted[frame=lines,fontsize=\footnotesize,linenos]{vhdl}{CodeFiles/EntidadInterfazEntrada.vhd}

	Como se puede ver en la Figura (\ref{fig:MotorPuertaOK}) el esquema obtenido una vez programado y sintetizado se corresponde con el que se pretendía.
    \begin{figure}[H]
		    \centering
		    \includegraphics[width = 0.6\textwidth ]{MotorPuertaOK}
		    \caption{Esquema exterior del simulador Motor Puerta}
		    \label{fig:MotorPuertaOK}
	\end{figure}
    Además podemos ver en la Figura (\ref{fig:MotorPuertaImplementacion}) como se compone internamente el bloque, como se codifica en hardware esta utilidad:
    \begin{figure}[H]
		    \centering
		    \includegraphics[width = 0.9\textwidth ]{MotorPuertaImplementacion}
		    \caption{Esquema interno del simulador Motor Puerta}
		    \label{fig:MotorPuertaImplementacion}
	\end{figure}
\subsection{Código Bloque \textit{Simulacion Motor Puerta (testbench)}:} \label{code:MotorPuerta_tb}
	\inputminted[frame=lines,fontsize=\footnotesize,linenos]{vhdl}{CodeFiles/EntidadInterfazEntrada_tb.vhd}

    Utilizando este testbench se obtiene el comportamiento que se puede ver en la Figura (\ref{fig:SimulacionMotorPuerta}):

    \begin{figure}[H]
		    \centering
		    \includegraphics[width = 0.8\textwidth ]{SimulacionMotorPuerta}
		    \caption{Simulación del testbench del simulador MotorPuerta}
		    \label{fig:SimulacionMotorPuerta}
	\end{figure}

\subsection{Entidad \textit{Interfaz Entrada}:} \label{code:InterfazEntrada}
	\inputminted[frame=lines,fontsize=\footnotesize,linenos]{vhdl}{CodeFiles/EntidadInterfazEntrada.vhd}

	Como se puede ver en la Figura (\ref{fig:EntidadInterfazEntradaOK}) el esquema obtenido una vez programado y sintetizado se corresponde con el que se pretendía.
    \begin{figure}[H]
		    \centering
		    \includegraphics[width = 0.6\textwidth ]{EntidadInterfazEntradaOK}
		    \caption{Esquema exterior de la Entidad InterfazEntrada}
		    \label{fig:EntidadInterfazEntradaOK}
	\end{figure}
    Además podemos ver en la Figura (\ref{fig:EntidadInterfazEntradaImplementacion}) como se compone internamente el bloque, como se codifica en hardware esta utilidad:
    \begin{figure}[H]
		    \centering
		    \includegraphics[width = 0.9\textwidth ]{EntidadInterfazEntradaImplementacion}
		    \caption{Esquema interno de la Entidad InterfazEntrada}
		    \label{fig:EntidadInterfazEntradaImplementacion}
	\end{figure}

\subsection{Código Entidad \textit{Interfaz Entrada (testbench)}:} \label{code:InterfazEntrada_tb}
	\inputminted[frame=lines,fontsize=\footnotesize,linenos]{vhdl}{CodeFiles/EntidadInterfazEntrada_tb.vhd}

    Utilizando este testbench se obtiene el comportamiento que se puede ver en la Figura (\ref{fig:SimulacionEntidadInterfazEntrada}):

    \begin{figure}[H]
		    \centering
		    \includegraphics[width = 0.8\textwidth ]{SimulacionEntidadInterfazEntrada}
		    \caption{Simulación del testbench de la entidad InterfazEntrada}
		    \label{fig:SimulacionEntidadInterfazEntrada}
	\end{figure}

\subsection{Código Entidad \textit{Control Ascensor}:} \label{code:ControlAscensor}
	\inputminted[frame=lines,fontsize=\footnotesize,linenos]{vhdl}{CodeFiles/EntidadControlAscensor.vhd}

	Como se puede ver en la Figura (\ref{fig:EntidadControlAscensorOK}) el esquema obtenido una vez programado y sintetizado se corresponde con el que se pretendía.
    \begin{figure}[H]
		    \centering
		    \includegraphics[width = 0.6\textwidth ]{EntidadControlAscensorOK}
		    \caption{Esquema exterior de la Entidad ControlAscensor}
		    \label{fig:EntidadControlAscensorOK}
	\end{figure}
    Además podemos ver en la Figura (\ref{fig:EntidadControlAscensorImplementacion}) como se compone internamente el bloque, como se codifica en hardware esta utilidad:
    \begin{figure}[H]
		    \centering
		    \includegraphics[width = 0.9\textwidth ]{EntidadControlAscensorImplementacion}
		    \caption{Esquema interno de la Entidad ControlAscensor}
		    \label{fig:EntidadControlAscensorImplementacion}
	\end{figure}

\subsection{Código Entidad \textit{Control Ascensor (testbench)}:} \label{code:ControlAscensor_tb}
	\inputminted[frame=lines,fontsize=\footnotesize,linenos]{vhdl}{CodeFiles/EntidadControlAscensor_tb.vhd}

    Utilizando este testbench se obtiene el comportamiento que se puede ver en la Figura (\ref{fig:SimulacionEntidadControlAscensor}):

    \begin{figure}[H]
		    \centering
		    \includegraphics[width = 0.8\textwidth ]{SimulacionEntidadControlAscensor}
		    \caption{Simulación del testbench de la entidad ControlAscensor}
		    \label{fig:SimulacionEntidadControlAscensor}
	\end{figure}

\subsection{Código Entidad \textit{Visualizacion}:} \label{code:Visualizacion}

\subsection{Código Código Entidad \textit{Acensor}:} \label{code:Acensor}

\subsection{Código Código Entidad \textit{Acensor (testbench)}:} \label{code:Acensor_tb}
 %importa el fichero Codigo.tex
    
    \section{Pruebas en placa y análisis de resultados:} \label{section:PruebasYResultados}

    	

	\begin{table}[H]
    \centering
		\begin{tabular}{|c|c|c||c|c|c|}
			\hline
			\rowcolor[rgb]{0.21,0.69,0.87}\multicolumn{6}{|c|}{  \textbf{ {Configuración Pines de entrada}}} \\
			\hline \hline
			\multicolumn{3}{|c|}{  \textbf{ {SensorVoy (4bits)}}} & \multicolumn{3}{|c|}{\textbf{SensorEstoy (4bits)}} \\
			\hline
			SensorVoy[0] & SW0 & F12 & SensorEstoy[0] & BTN0 & M13 \\
			SensorVoy[1] & SW1 & G12 & SensorEstoy[1] & BTN1 & M14 \\
			SensorVoy[2] & SW2 & H14 & SensorEstoy[2] & BTN2 & L13 \\
			SensorVoy[3] & SW3 & H13 & SensorEstoy[3] & BTN3 & L14 \\
			\hline
		\end{tabular}
		\caption{ Configuración de los pines de entrada con las entradas del sistema }
		\label{tab:pinEntradas}
	\end{table}

	\begin{table}[H]
    \centering
		\begin{tabular}{|c|c|}
			\hline
			\rowcolor[rgb]{0.21,0.69,0.87}\multicolumn{2}{|c|}{  \textbf{ {Configuración pin CLK}}} \\
			\hline \hline
			CLK & T9 \\ 
			\hline
		\end{tabular}
		\caption{ Configuración del pin para el CLK }
		\label{tab:pinCLK}
	\end{table}

	\begin{table}[H]
    \centering
		\begin{tabular}{|c|c||c|c|c|}
			\hline
			\rowcolor[rgb]{0.21,0.69,0.87}\multicolumn{4}{|c|}{  \textbf{ {Configuración Pines de los displays de 7 segmentos}}} \\
			\hline \hline
			\multicolumn{2}{|c|}{  \textbf{ { Salida7s (7bits)}}} & \multicolumn{2}{|c|}{\textbf{Ánodos de Control}} \\
			\hline
			Salida7s[0] & E14 & Motor & AN0 & D14 \\
			Salida7s[1] & G13 & Puerta & AN1 & G14 \\
			Salida7s[2] & N15 & PisoVoy & AN2 & F14 \\
			Salida7s[3] & P15 & PisoEstoy & AN3 & E13 \\
			Salida7s[4] & R16 & - & - & - \\
			Salida7s[5] & F13 & - & - & - \\
			Salida7s[6] & N16 & - & - & - \\
			\hline
		\end{tabular}
		\caption{ Configuración de los pines de salida al display de 7 segmentos }
		\label{tab:pin7s}
	\end{table} %importa el fichero PruebasYResultados.tex
    
    \begin{appendices}
    \newpage
    \section{Codificación Entradas-Salidas} \label{app:codEntSal}
    
        \begin{table}[H]
        \centering
			\begin{tabular}{|c|c|c|c|}
				\hline
				\rowcolor[rgb]{0.21,0.69,0.87}\multicolumn{4}{|c|}{  \textbf{ {Codificación Pisos}}} \\
				\hline \hline
				 & \textbf{  PisoVoy (pulsadores)  } & \textbf{  PisoEstoy (finalCarrera)  } & \textbf{  7 Segmentos  }  \\
				\hline
				Piso1 & 0001 & 0001 & Imagen \\
				\hline
				Piso2 & 0010 & 0010 & Imagen \\
				\hline
				Piso3 & 0100 & 0100 & Imagen \\
				\hline
				Piso4 & 1000 & 1000 & Imagen \\
				\hline
				 
			\end{tabular}
			\caption{ Codificación para los diferentes pisos y su traducción  al número de piso y al display de 7 segmentos }
			\label{tab:tabla1ApendiceA}
		\end{table}
		
		
        \begin{table}[H]
        \centering
			\begin{tabular}{|c|c|}
				\hline
				\rowcolor[rgb]{0.21,0.69,0.87}\multicolumn{2}{|c|}{  \textbf{ {Funcionamiento Motor}}} \\
				\hline \hline
				 & \textbf{ Codificación Interna }  \\
				\hline
				Motor Parado & 00  \\
				\hline
				Motor girando en sentido de Subida & 01  \\
				\hline
				Motor girando en sentido de Bajada & 10 \\ 
				\hline
				 
			\end{tabular}
			\caption{ Codificación para los diferentes pisos y su traducción  al número de piso y al display de 7 segmentos }
			\label{tab:tabla2ApendiceA}
		\end{table}

		
	\section{Codificación para display de 7 segmentos}	\label{app:7segmentos}
		\begin{table}[H]
        \centering
			\begin{tabular}{|ccccc|}
				\hline
				\rowcolor[rgb]{0.21,0.69,0.87}\multicolumn{5}{|c|}{  \textbf{ {Caracteres en binario para display de 7 segmentos}}} \\
				\hline \hline
				\multicolumn{3}{|c|}{  \textbf{ {Funcionamiento Motor}}} & \multicolumn{2}{|c|}{\textbf{Funcionamiento Puerta}} \\
				\hline
				 \includegraphics[width = 0.15\textwidth ]{CodCaracteres7s/subida} &
				 \includegraphics[width = 0.15\textwidth ]{CodCaracteres7s/bajada}  &
				 \includegraphics[width = 0.15\textwidth ]{CodCaracteres7s/parado} &
				 \includegraphics[width = 0.15\textwidth ]{CodCaracteres7s/Cerrada}  &
				 \includegraphics[width = 0.15\textwidth ]{CodCaracteres7s/abierta}  \\
				 0100100 & 0000000 & 1111110 & 0110001 & 0001000 \\ 	
				\hline
				 \includegraphics[width = 0.15\textwidth ]{CodCaracteres7s/p1} &
				 \includegraphics[width = 0.15\textwidth ]{CodCaracteres7s/p2}  &
				 \includegraphics[width = 0.15\textwidth ]{CodCaracteres7s/p3} &
				 \includegraphics[width = 0.15\textwidth ]{CodCaracteres7s/p4}  &
				 \includegraphics[width = 0.15\textwidth ]{CodCaracteres7s/error}  \\
				 1001111 & 0010010 & 0000110 & 1001100 & 0110000 \\ 
				\hline
			\end{tabular}
			\caption{ Codificación en binario para mostrar la información en el display de 7 segmentos }
			\label{tab:tabla1ApendiceB}
		\end{table}
	\newpage	
\end{appendices} %importa el fichero Apendices.tex

    \newpage
    \addcontentsline{toc}{section}{Índice de figuras} % para que aparezca en el indice de contenidos
    \listoffigures
    \addcontentsline{toc}{section}{Índice de tablas} % para que aparezca en el indice de contenidos
	\listoftables
	
\end{document}
