\subsection{Código Bloque \textit{Decodificador a 7 segmentos}:} \label{code:Decodificador7s}
	\inputminted[frame=lines,fontsize=\footnotesize,linenos]{vhdl}{CodeFiles/Decodificador7s.vhd}

\subsection{Código Bloque \textit{Decodificador a 7 segmentos (testbench)}:} \label{code:Decodificador7s_tb}
	
	\inputminted[frame=lines,fontsize=\footnotesize,linenos]{vhdl}{CodeFiles/Decodificador7s_tb.vhd}
	
	Utilizando este testbench se obtiene el comportamiento que se puede ver en la Figura (\ref{fig:SimulacionDecodificador7s}):

    \begin{figure}[H]
		    \centering
		    \includegraphics[width = 0.8\textwidth ]{SimulacionDecodificador7s}
		    \caption{Simulación del testbench del bloque Decodificador7s}
		    \label{fig:SimulacionDecodificador7s}
	\end{figure}

\subsection{Código Bloque \textit{Divisor de frecuencia}:} \label{code:DivisorFrecuencia}

\subsection{Código Bloque \textit{Divisor de frecuencia (testbench)}:} \label{code:DivisorFrecuencia_Tb}

\subsection{Código Bloque \textit{PistoActual}:} \label{code:PisoActual}
    \inputminted[frame=lines,fontsize=\footnotesize,linenos]{vhdl}{CodeFiles/PisoActual.vhd}
    
    Como se puede ver en la Figura (\ref{fig:BloquePisoActualOK}) el esquema obtenido una vez programado y sintetizado se corresponde con el que se pretendía.
    \begin{figure}[H]
		    \centering
		    \includegraphics[width = 0.6\textwidth ]{BloquePisoActualOK}
		    \caption{Esquema exterior del bloque Piso Actual}
		    \label{fig:BloquePisoActualOK}
	\end{figure}
    Además podemos ver en la Figura (\ref{fig:BloquePisoActualImplementacion}) como se compone internamente el bloque, como se codifica en hardware esta utilidad:
    \begin{figure}[H]
		    \centering
		    \includegraphics[width = 0.9\textwidth ]{BloquePisoActualImplementacion}
		    \caption{Esquema interno del bloque Piso Actual}
		    \label{fig:BloquePisoActualImplementacion}
	\end{figure}

\subsection{Código Bloque \textit{PistoActual (testbench)}:} \label{code:PisoActual_tb}
    \inputminted[frame=lines,fontsize=\footnotesize,linenos]{vhdl}{CodeFiles/PisoActual_tb.vhd}
    
    Utilizando este testbench se obtiene el comportamiento que se puede ver en la Figura (\ref{fig:SimulacionPisoActual}):

    \begin{figure}[H]
		    \centering
		    \includegraphics[width = 0.8\textwidth ]{SimulacionPisoActual}
		    \caption{Simulación del testbench del bloque PisoActual}
		    \label{fig:SimulacionPisoActual}
	\end{figure}


\subsection{Código Bloque \textit{Bloqueador PisoVoy}:} \label{code:BloqueadorpisoVoy}	
    \inputminted[frame=lines,fontsize=\footnotesize,linenos]{vhdl}{CodeFiles/BloqueadorPisoVoy.vhd}
	Como se puede ver en la Figura (\ref{fig:BloqueBloqueadorPisoVoyOK}) el esquema obtenido una vez programado y sintetizado se corresponde con el que se pretendía.
    \begin{figure}[H]
		    \centering
		    \includegraphics[width = 0.6\textwidth ]{BloqueBloqueadorPisoVoyOK}
		    \caption{Esquema exterior del bloque Bloqueador PisoVoy}
		    \label{fig:BloqueBloqueadorPisoVoyOK}
	\end{figure}
    Además podemos ver en la Figura (\ref{fig:BloqueBloqueadorPisoVoyImplementacion}) como se compone internamente el bloque, como se codifica en hardware esta utilidad:
    \begin{figure}[H]
		    \centering
		    \includegraphics[width = 0.9\textwidth ]{BloqueBloqueadorPisoVoyImplementacion}
		    \caption{Esquema interno del bloque Bloqueador PisoVoy}
		    \label{fig:BloqueBloqueadorPisoVoyImplementacion}
	\end{figure}

\subsection{Código Bloque \textit{Bloqueador PisoVoy (testbench)}:} \label{code:BloqueadorpisoVoy_tb}
    \inputminted[frame=lines,fontsize=\footnotesize,linenos]{vhdl}{CodeFiles/BloqueadorPisoVoy_tb.vhd}

    Utilizando este testbench se obtiene el comportamiento que se puede ver en la Figura (\ref{fig:SimulacionBloqueadorPisoVoy}):

    \begin{figure}[H]
		    \centering
		    \includegraphics[width = 0.8\textwidth ]{SimulacionBloqueadorPisoVoy}
		    \caption{Simulación del testbench del bloque BloqueadorPisoVoy}
		    \label{fig:SimulacionBloqueadorPisoVoy}
	\end{figure}

\subsection{Código Bloque \textit{Decodificador Binario a Entero}:} \label{code:DecodificadorBinarioEntero}
    \inputminted[frame=lines,fontsize=\footnotesize,linenos]{vhdl}{CodeFiles/DecodificadorBinarioEntero.vhd}
    
    Como se puede ver en la Figura (\ref{fig:BloqueDecodificadorBinarioEnteroOK}) el esquema obtenido una vez programado y sintetizado se corresponde con el que se pretendía.
    \begin{figure}[H]
		    \centering
		    \includegraphics[width = 0.6\textwidth ]{BloqueDecodificadorBinarioEnteroOK}
		    \caption{Esquema exterior del bloque Decodificador Binario-Entero}
		    \label{fig:BloqueDecodificadorBinarioEnteroOK}
	\end{figure}
    Además podemos ver en la Figura (\ref{fig:BloqueDecodificadorBinarioEnteroImplementacion}) como se compone internamente el bloque, como se codifica en hardware esta utilidad:
    \begin{figure}[H]
		    \centering
		    \includegraphics[width = 0.9\textwidth ]{BloqueDecodificadorBinarioEnteroImplementacion}
		    \caption{Esquema interno del bloque Decodificador Binario-Entero}
		    \label{fig:BloqueDecodificadorBinarioEnteroImplementacion}
	\end{figure}
    
\subsection{Código Bloque \textit{Decodificador Binario a Entero (testbench)}:} \label{code:DecodificadorBinarioEntero_tb}
    \inputminted[frame=lines,fontsize=\footnotesize,linenos]{vhdl}{CodeFiles/DecodificadorBinarioEntero_tb.vhd}

    Utilizando este testbench se obtiene el comportamiento que se puede ver en la Figura (\ref{fig:SimulacionDecodificadorBinarioEntero}):

    \begin{figure}[H]
		    \centering
		    \includegraphics[width = 0.8\textwidth ]{SimulacionDecodificadorBinarioEntero}
		    \caption{Simulación del testbench del bloque DecodificadorBinarioEntero}
		    \label{fig:SimulacionDecodificadorBinarioEntero}
	\end{figure}

\subsection{Código Bloque \textit{Comparador}:} \label{code:Comparador}
    \inputminted[frame=lines,fontsize=\footnotesize,linenos]{vhdl}{CodeFiles/Comparador.vhd}	

	Como se puede ver en la Figura (\ref{fig:BloqueComparadorOK}) el esquema obtenido una vez programado y sintetizado se corresponde con el que se pretendía.
    \begin{figure}[H]
		    \centering
		    \includegraphics[width = 0.6\textwidth ]{BloqueComparadorOK}
		    \caption{Esquema exterior del Bloque Comparador}
		    \label{fig:BloqueComparadorOK}
	\end{figure}
    Además podemos ver en la Figura (\ref{fig:BloqueComparadorImplementacion}) como se compone internamente el bloque, como se codifica en hardware esta utilidad:
    \begin{figure}[H]
		    \centering
		    \includegraphics[width = 0.9\textwidth ]{BloqueComparadorImplementacion}
		    \caption{Esquema interno del Bloque Comparador}
		    \label{fig:BloqueComparadorImplementacion}
	\end{figure}

\subsection{Código Bloque \textit{Comparador (testbench)}:} \label{code:Comparador_tb}
    \inputminted[frame=lines,fontsize=\footnotesize,linenos]{vhdl}{CodeFiles/Comparador_tb.vhd}

    Utilizando este testbench se obtiene el comportamiento que se puede ver en la Figura (\ref{fig:SimulacionComparador}):

    \begin{figure}[H]
		    \centering
		    \includegraphics[width = 0.8\textwidth ]{SimulacionComparador}
		    \caption{Simulación del testbench del bloque Comparador}
		    \label{fig:SimulacionComparador}
	\end{figure}

\subsection{Código Bloque \textit{ Simulacion Motor Puerta}:} \label{code:MotorPuerta}
	\inputminted[frame=lines,fontsize=\footnotesize,linenos]{vhdl}{CodeFiles/EntidadInterfazEntrada.vhd}

	Como se puede ver en la Figura (\ref{fig:MotorPuertaOK}) el esquema obtenido una vez programado y sintetizado se corresponde con el que se pretendía.
    \begin{figure}[H]
		    \centering
		    \includegraphics[width = 0.6\textwidth ]{MotorPuertaOK}
		    \caption{Esquema exterior del simulador Motor Puerta}
		    \label{fig:MotorPuertaOK}
	\end{figure}
    Además podemos ver en la Figura (\ref{fig:MotorPuertaImplementacion}) como se compone internamente el bloque, como se codifica en hardware esta utilidad:
    \begin{figure}[H]
		    \centering
		    \includegraphics[width = 0.9\textwidth ]{MotorPuertaImplementacion}
		    \caption{Esquema interno del simulador Motor Puerta}
		    \label{fig:MotorPuertaImplementacion}
	\end{figure}
\subsection{Código Bloque \textit{Simulacion Motor Puerta (testbench)}:} \label{code:MotorPuerta_tb}
	\inputminted[frame=lines,fontsize=\footnotesize,linenos]{vhdl}{CodeFiles/EntidadInterfazEntrada_tb.vhd}

    Utilizando este testbench se obtiene el comportamiento que se puede ver en la Figura (\ref{fig:SimulacionMotorPuerta}):

    \begin{figure}[H]
		    \centering
		    \includegraphics[width = 0.8\textwidth ]{SimulacionMotorPuerta}
		    \caption{Simulación del testbench del simulador MotorPuerta}
		    \label{fig:SimulacionMotorPuerta}
	\end{figure}

\subsection{Entidad \textit{Interfaz Entrada}:} \label{code:InterfazEntrada}
	\inputminted[frame=lines,fontsize=\footnotesize,linenos]{vhdl}{CodeFiles/EntidadInterfazEntrada.vhd}

	Como se puede ver en la Figura (\ref{fig:EntidadInterfazEntradaOK}) el esquema obtenido una vez programado y sintetizado se corresponde con el que se pretendía.
    \begin{figure}[H]
		    \centering
		    \includegraphics[width = 0.6\textwidth ]{EntidadInterfazEntradaOK}
		    \caption{Esquema exterior de la Entidad InterfazEntrada}
		    \label{fig:EntidadInterfazEntradaOK}
	\end{figure}
    Además podemos ver en la Figura (\ref{fig:EntidadInterfazEntradaImplementacion}) como se compone internamente el bloque, como se codifica en hardware esta utilidad:
    \begin{figure}[H]
		    \centering
		    \includegraphics[width = 0.9\textwidth ]{EntidadInterfazEntradaImplementacion}
		    \caption{Esquema interno de la Entidad InterfazEntrada}
		    \label{fig:EntidadInterfazEntradaImplementacion}
	\end{figure}

\subsection{Código Entidad \textit{Interfaz Entrada (testbench)}:} \label{code:InterfazEntrada_tb}
	\inputminted[frame=lines,fontsize=\footnotesize,linenos]{vhdl}{CodeFiles/EntidadInterfazEntrada_tb.vhd}

    Utilizando este testbench se obtiene el comportamiento que se puede ver en la Figura (\ref{fig:SimulacionEntidadInterfazEntrada}):

    \begin{figure}[H]
		    \centering
		    \includegraphics[width = 0.8\textwidth ]{SimulacionEntidadInterfazEntrada}
		    \caption{Simulación del testbench de la entidad InterfazEntrada}
		    \label{fig:SimulacionEntidadInterfazEntrada}
	\end{figure}

\subsection{Código Entidad \textit{Control Ascensor}:} \label{code:ControlAscensor}
	\inputminted[frame=lines,fontsize=\footnotesize,linenos]{vhdl}{CodeFiles/EntidadControlAscensor.vhd}

	Como se puede ver en la Figura (\ref{fig:EntidadControlAscensorOK}) el esquema obtenido una vez programado y sintetizado se corresponde con el que se pretendía.
    \begin{figure}[H]
		    \centering
		    \includegraphics[width = 0.6\textwidth ]{EntidadControlAscensorOK}
		    \caption{Esquema exterior de la Entidad ControlAscensor}
		    \label{fig:EntidadControlAscensorOK}
	\end{figure}
    Además podemos ver en la Figura (\ref{fig:EntidadControlAscensorImplementacion}) como se compone internamente el bloque, como se codifica en hardware esta utilidad:
    \begin{figure}[H]
		    \centering
		    \includegraphics[width = 0.9\textwidth ]{EntidadControlAscensorImplementacion}
		    \caption{Esquema interno de la Entidad ControlAscensor}
		    \label{fig:EntidadControlAscensorImplementacion}
	\end{figure}

\subsection{Código Entidad \textit{Control Ascensor (testbench)}:} \label{code:ControlAscensor_tb}
	\inputminted[frame=lines,fontsize=\footnotesize,linenos]{vhdl}{CodeFiles/EntidadControlAscensor_tb.vhd}

    Utilizando este testbench se obtiene el comportamiento que se puede ver en la Figura (\ref{fig:SimulacionEntidadControlAscensor}):

    \begin{figure}[H]
		    \centering
		    \includegraphics[width = 0.8\textwidth ]{SimulacionEntidadControlAscensor}
		    \caption{Simulación del testbench de la entidad ControlAscensor}
		    \label{fig:SimulacionEntidadControlAscensor}
	\end{figure}

\subsection{Código Entidad \textit{Visualizacion}:} \label{code:Visualizacion}

\subsection{Código Código Entidad \textit{Acensor}:} \label{code:Acensor}

\subsection{Código Código Entidad \textit{Acensor (testbench)}:} \label{code:Acensor_tb}
