\subsection{Entidad \textit{Acensor}:}
    
    La interfaz, entradas y salidas, de este bloque se puede ver su representación en la Figura \ref{fig:EntidadIAscensor}:
    
    \begin{figure}[H]
		    \centering
		    \includegraphics[width = 0.8\textwidth ]{EntidadAscensor}
		    \caption{Diagrama de la Interfaz de la Entidad Ascensor}
		    \label{fig:EntidadIAscensor}
	\end{figure}
	
	Esta es una entidad de alto nivel que encapsula todo el funcionamiento del ascensor. Se puede ver el diagrama de bloques interno de esta entidad en la siguiente figura: 
	
	\begin{figure}[H]
		    \centering
		    \includegraphics[width = .85\textwidth ]{EntidadAscensorEntidades}
		    \caption{Diagrama interno de la entidad Ascensor}
		    \label{fig:EntidadIAscensorEntidades}
	\end{figure}
	
	Como se puede ver en la figura anterior se han encapsulado los bloques en diferentes entidades en función de su finalidad, estas entidades se pueden ver a continuación. En las representaciones de estas entidades se verá que hay ciertos bloques repetidos, se trata del mismo bloque utilizado dos veces, no de bloques diferentes, que se han representado por duplicado para mayor claridad. \\ 

	Se adelanta el esquema del funcionamiento general del ascensor (obviando las entidades generales):

	\begin{figure}[H]
		    \centering
		    \includegraphics[width = .85\textwidth ]{EntidadAscensorInterior}
		    \caption{Diagrama funcionamiento del Ascensor}
		    \label{fig:EntidadIAscensorInterior}
	\end{figure}

\subsection{Entidad \textit{Interfaz Entrada}:}
	Esta entidad se encarga de gestionar los datos de las entradas para adaptarlos al funcionamiento interno de nuestro sistema, en este caso se encarga de codificar y modular correctamente las señales de los sensores del piso al que voy y del piso en que estoy. \\ 

	Como se ve en la figura (\ref{fig:EntidadesAscensorE1}), engloba dos bloques, que se desgran en los siguientes apartados.
	\begin{figure}[H]
		    \centering
		    \includegraphics[width = .85\textwidth ]{EntidadAscensorInterior(E1)}
		    \caption{Representación de la Entidad Interfaz Entrada}
		    \label{fig:EntidadesAscensorE1}
	\end{figure}

\subsection{Entidad \textit{Control Ascensor}:}
	Esta entidad se encarga de gestionar el funcionamiento del ascensor propiamente dicho comparando las lecturas de los sensores y decidiendo que deben hacer los actuadores, el motor y la puerta.  \\ 

	Como se ve en la figura (\ref{fig:EntidadesAscensorE3}), engloba tres bloques, que se explican más adelante.
	\begin{figure}[H]
		    \centering
		    \includegraphics[width = .85\textwidth ]{EntidadAscensorInterior(E3)}
		    \caption{Representación de la Entidad Control Ascensor}
		    \label{fig:EntidadesAscensorE3}
	\end{figure}

	Como se puede observar el bloque Codigicador Binario-Decimal se utiliza dos veces.

\subsection{Entidad \textit{Visualizacion}:}	
	Esta entidad se encarga de gestionar el funcionamiento del ascensor propiamente dicho comparando las lecturas de los sensores y decidiendo que deben hacer los actuadores, el motor y la puerta.  \\ 

	Como se ve en la figura (\ref{fig:EntidadesAscensorE2}), engloba cuatro bloques bloques, que se detallan en apartados posteriores.
	\begin{figure}[H]
		    \centering
		    \includegraphics[width = .85\textwidth ]{EntidadAscensorInterior(E2)}
		    \caption{Representación de la Entidad Visualizacion}
		    \label{fig:EntidadesAscensorE2}
	\end{figure}

	Como se puede observar el bloque Decodificador7s y el bloque DivisorFrecuencia se utilizan dos veces cada uno.


\subsection{Bloque \textit{Decodificador a 7 segmentos}:}
    Este bloque es el encargado de traducir el piso en el que se encuentra el ascensor y el piso objetivo para poder mostrarlo en el display de 7 segmentos. Para ello es importante recordar el Cuadro (\ref{tab:tabla1ApendiceA}) del Apéndice \ref{app:codEntSal} donde podemos ver como se codifica internamente el número de piso. \\ 
    
    La salida que se verá en el display para cada caso se puede ver en la figura siguiente:
    
    \begin{figure}[H]
		    \centering
		    \includegraphics[width = 1\textwidth ]{displays7s}
		    \caption{Salida en los displays de 7 segmentos}
		    \label{fig:displays7s}
	\end{figure}
    
    La interfaz, entradas y salidas, de este bloque se puede ver su representación en la Figura \ref{fig:BloqueDecodificador7seg}:
    
    \begin{figure}[H]
		    \centering
		    \includegraphics[width = 0.7\textwidth ]{BloqueDecodificador}
		    \caption{Diagrama Interfaz Bloque Decodificador a 7 segmentos}
		    \label{fig:BloqueDecodificador7seg}
	\end{figure}
\subsection{Bloque \textit{Divisor de frecuencia}:}

\subsection{Bloque \textit{PistoActual}:}
    Como se ha dicho anteriormente en cada piso hay un final de carrera que detecta el paso del ascensor. El prpósito de este bloque es el de filtrar dicha entrada de 4 bits. En este caso lo que interesa es saber en que piso estoy o en que piso he estado por última vez. Cuando el ascensor se encuentra entre dos pisos la entrada de los sensores será \textit{0000}, este bloque lo que hará será mantener en la salida del mismo el último piso por el que haya pasado el ascensor. \\ 
    
    Como se puede apreciar en el siguiente diagrama este bloque tiene una entrada, un vector de 4 bits (los finales de carrera de cada piso) y una salida, también de 4 bits, codificando el piso en el que se encuentra actualmente. \\ 
    
    Se puede consultar dicha codificación en el Cuadro (\ref{tab:tabla1ApendiceA}) del Apéndice \ref{app:codEntSal}. \\ 
    
    La interfaz, entradas y salidas, de este bloque se puede ver su representación en la Figura \ref{fig:BloquePisoActual}:
    
    \begin{figure}[H]
		    \centering
		    \hspace*{-1.8cm}
		    \includegraphics[width = 0.6\textwidth ]{BloquePisoActual}
		    \caption{Diagrama Bloque PisoActual}
		    \label{fig:BloquePisoActual}
	\end{figure}
	
	Se puede consultar el código VHDL de este módulo en el Apartado \ref{code:PisoActual} así como el código de su testbench correspondiente en el Apartado \ref{code:PisoActual_tb}.

\subsection{Bloque \textit{Bloqueador PisoVoy}:}
    
    La interfaz, entradas y salidas, de este bloque se puede ver su representación en la Figura \ref{fig:BloqueBloqueadorPisoVoy}:
    
    
    \begin{figure}[H]
		    \centering
		    \hspace*{-1.8cm}
		    \includegraphics[width = 0.6\textwidth ]{BloqueBloqueadorPisoVoy}
		    \caption{Diagrama Bloque Bloqueador PisoVoy}
		    \label{fig:BloqueBloqueadorPisoVoy}
	\end{figure}
	
\subsection{Código Bloque \textit{Decodificador Binario a Decimal}:} 
	En este bloque se traduce la señal binaria que codifica tanto el piso actual como el piso de destino a decimal para su posterior comparación.
	Se puede consultar dicha codificación en el Cuadro (\ref{tab:tabla1ApendiceA}) del Apéndice \ref{app:codEntSal}. \\ 
	
\subsection{Bloque \textit{Comparador}:}

    La interfaz, entradas y salidas, de este bloque se puede ver su representación en la Figura \ref{fig:BloqueComparador}:
    
    \begin{figure}[H]
		    \centering
		    \includegraphics[width = 0.6\textwidth ]{BloqueComparador}
		    \caption{Diagrama Bloque Comparador}
		    \label{fig:BloqueComparador}
	\end{figure}



\subsection{Bloque \textit{Controlador Motor} y \textit{Controlador Puerta}:}
    El bloque Controlador Motor se encarga de controlar el motor. En la entrada recibe dos bits con la codificación que se muestra en el Cuadro (\ref{tab:tabla2ApendiceA}) del Apéndice \ref{app:codEntSal} y los gestiona para modelar y simular con los recursos de la placa el movimiento del mismo.

    El bloque Controlador Puerta se encarga del control de la puerta del ascensor. En la entrada recibe un bit con la codificación que se muestra en el Cuadro (\ref{tab:tabla3ApendiceA}) del Apéndice \ref{app:codEntSal} que indica si la puerta está abierta o cerrada; este bloque actua como interfaz entre esa codificación y el modelo real, en este caso se simulará con los recursos disponibles en la placa.

    En este caso no tenemos ni puerta ni motor físico así que se ha decidido sacar la "reacción" que tendrían estos elementos por dos de los display de 7 segmentos de que se dispone, de esta forma se podrá ver en los displays los siguientes caracteres:

	\begin{table}[H]
        \centering
			\begin{tabular}{|c|c||c|c|}
				\hline
				\rowcolor[rgb]{0.21,0.69,0.87}\multicolumn{2}{|c|}{  \textbf{ {Funcionamiento Motor}}} & \multicolumn{2}{|c|}{  \textbf{ {Funcionamiento Puerta}}} \\
				\hline \hline
				 & \textbf{ Codificación Interna }  & \textbf{ Codificación Interna }  \\
				\hline
				Subiendo & \includegraphics[width = 0.8\textwidth ]{CodCaracteres7s/subida}} & Abierta & \includegraphics[width = 0.8\textwidth ]{CodCaracteres7s/abierta}}  \\
				\hline
				Bajando & \includegraphics[width = 0.8\textwidth ]{CodCaracteres7s/bajada}} & Cerrada & \includegraphics[width = 0.8\textwidth ]{CodCaracteres7s/cerrada}}  \\
				\hline
				Parado & \includegraphics[width = 0.8\textwidth ]{parado}} & - & - \\
				\hline				 
			\end{tabular}
			\caption{ Salida en displays 7 segmentos para el control del motor y de la puerta }
			\label{tab:tabla2ApendiceA}
	\end{table}
    
	Este bloque consta de un bloque para la codificación de ambos displays en serie con los respectivos divisores de frecuencia para cada display.